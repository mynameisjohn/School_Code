\documentclass[a4paper,12pt]{article}
\parindent 0pt
\parskip 1mm
\usepackage{amsmath}
\usepackage[dvips]{epsfig}

\begin{document}

\begin{center}

{\Large\bf MA 555 - Numerical Analysis}

\bigskip

{\large\bf Assignment \# 2}
\smallskip

{\large\bf John Joseph}
\end{center}

{\bf Problem 1}
\smallskip

Given the function $f(x)=\frac{1}{1+x}$, we can find an osculating parabola $y_p(x)$ that mimics it with some degree of error by solving the equation

\begin{equation*}
e(x)=f(x)-y_p(x)
\end{equation*}

If we take $y_p(x)$ as a second order polynomial, we must solve the following equations at $x=0$ :
\begin{eqnarray*}
y_p(x)=a_0+a_1x+a_2x^2 \\
e(0)=0 =f(0)-y_p(0)=1-a_0\\
a_0=1 \\
f'(x)=\frac{-1}{(1+x)^2} \\
e'(0)=0 =f'(0)-y'_p(0) =-1-a_1\\
a_1=-1 \\
f''(x)=\frac{2}{(1+x)^3} \\
e''(0)=0 =f''(0)-y''_p(0) =2-2a_2\\
a_2=1\\
y_p(x)=1-x+x^2\\
e(x)=\frac{1}{1+x}-(1-x+x^2)
\end{eqnarray*}

$e(x)$ is our error function, and we can use Taylor's Theorem to solve for its lower and upper bounds. By Taylor's Theorem,

\begin{equation*}
e(x) =x^3 \Big( \frac{1}{3!} e'''(\xi) \Big), 0<\xi<x, e'''(0) \neq 0
\end{equation*}

We can use this fact to find the asymptoticerror  value as $x \to 0$ by dividing out the $x^3$ term as shown in class (note that any other
$x^n$ would cause the limit to either diverge or go to 0). 
Solving the following limit:

\begin{equation*}
\lim_{x \to 0} \frac{x^3 \Big( \frac{1}{3!} e'''(\xi) \Big)}{x^3} = \lim_{x \to 0} \frac{\frac{1}{1+x}-(1-x+x^2)}{x^3} = -1
\end{equation*}

This limit was solved in Wolfram, and shows that the error asymptotically approaces $-x^3$, and that the error is $O(x^3)$. 
\bigskip

{\bf Problem 2}
\smallskip

Given 
\begin{equation*}
r(x)=(1+x)^{\frac{1}{2}}-(a_0+a_1x+a_2x^2+a_3x^3)
\end{equation*}

Similarly to problem 1, we set $r(0)=r'(0)=r''(0)=r'''(0)=0$ to get a third order polynomial that approximates our function. 


\begin{eqnarray*}
r(0)=0 =f(0)-y_p(0)=1-a_0\\
a_0=1 \\
f'(x)=\frac{1}{2}(1+x)^{-\frac{1}{2}} \\
r'(0)=0=f'(0)-y'_p(0) =\frac{1}{2}-a_1\\
a_1=\frac{1}{2} \\
f''(x)=-\frac{1}{4}(1+x)^{-\frac{3}{2}} \\
r''(0)=0 =f''(0)-y''_p(0) =-\frac{1}{4}-2a_2\\
a_2=-\frac{1}{8}\\
f'''(x)=\frac{3}{8}(1+x)^{-\frac{5}{2}} \\
r'''(0)=0 =f'''(0)-y'''_p(0) =\frac{3}{8}-6a_3\\
a_3=\frac{1}{16}\\
r(x)=(1+x)^{\frac{1}{2}}-(1+\frac{1}{2}x-\frac{1}{8}x^2+\frac{1}{16}x^3)
\end{eqnarray*}

Again, by Taylor's Theorem:

\begin{equation*}
r^{(4)}(x) \neq 0, r(x)=\frac{1}{4!}x^4r^{(4)}(\xi),   0<\xi<x
\end{equation*}

If we take the limit as $x \rightarrow 0$, we see that the error function $r(x)$ asymptotically approaches

\begin{equation*}
\lim_{x \to 0} \frac{\frac{1}{4!}x^4r^{(4)}(\xi)}{x^4} = \frac{1}{4!}r^{(4)}(0)
\end{equation*}


{\bf Problem  4}
\smallskip

We want to fix the variables $a$ and $b$ so that the second order terms of the Taylor polynomial of the function $e(x)$ equals 0

\begin{equation*}
e(x)=ln(1+x)-\frac{ax}{1+bx}
\end{equation*}


\begin{eqnarray*}
e(0)=0 =0-0\\
e'(0)=0=\frac{1}{(1+x)}-\frac{a}{(bx+1)^2} \Big |_{x=0}\\
a=1 \\
e''(0)=0=-\frac{1}{(1+x)^2}+\frac{2b}{(bx+1)^3} \Big |_{x=0}\\
b=\frac{1}{2} \\
e(x)=ln(1+x)-\frac{x}{x+\frac{1}{2}} = ln(1+x)-\frac{2x}{2+x}
\end{eqnarray*}

Using Taylor's theorem, we also know that 

\begin{equation*}
e(x)=\frac{1}{3!}x^3e'''(\xi),  0<\xi<x
\end{equation*}

This is our error function,and by looking at it we can see that we must divide it by $x^3$ to get a non-zero / non-diverging value. We can evaluate the error  asymptotically by taking the limit

\begin{equation*}
\lim_{x \to 0} \frac{\frac{1}{3!}x^3e'''(\xi)}{x^3} = \lim_{x \to 0} \frac{ln(1+x)-\frac{2x}{2+x}}{x^3}  = \frac{1}{12}
\end{equation*}

Which was solved using Wolfram. This shows that our error goes asymptotically to $\frac{1}{12}x^3$ as $x \to 0$, so our error is $O(x^3)$. 

\vfil\eject

{\bf Problem  3}
\smallskip

We are solving the IVP 

\begin{equation*}
y'(x)=x^2+y^2, y(1)=1
\end{equation*}

And are asked to approximating with an osculating cubic Taylor polynomial centered on $x=1$:
\begin{equation*}
y_c(x)=y(1)+(x-1)y'(1)+\frac{1}{2!}(x-1)^2y''(1)+\frac{1}{3!}(x-1)^3y'''(1)
\end{equation*}

We do so as follows:

\begin{eqnarray*}
y(1) = 1  \\
y'(1)=x^2+y^2 \Big|_{x=1}=2 \\
y''(x)=2x+2y(x)y'(x) \Big |_{x=1} = 6 \\
y'''(x)=2+2[y'(x)^2+y(x)y''(x)] \Big |_{x=1} = 22 \\
y_c(x)=1+2(x-1)+\frac{6}{2!}(x-1)^2+\frac{22}{3!}(x-1)^3
\end{eqnarray*}

{\bf Problem  5}
\smallskip

Using the following Python script, I found that $n=31$. 

\begin{verbatim}
import math

x=7.5
y=1.0
n=0
while (math.exp(x)-y) > 1e-7:
     n=n+1
     y=y+(1.0/math.factorial(n))*x**n
print n

...

31
\end{verbatim}

The error of the polynomial when n=31 is 4.928e-08, which is less than 1e-07 (or 0.0000001). 30 iterations yields an error value of 2.121e-07, which is greater
than 1e-07.  

\end{document}
