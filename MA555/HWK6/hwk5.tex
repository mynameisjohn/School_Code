\documentclass[a4paper,12pt]{article}
\parindent 0pt
\parskip 1mm
\usepackage{amsmath}
\usepackage[dvips]{epsfig}

\begin{document}

\begin{center}

{\Large\bf MA 555 - Numerical Analysis}

\bigskip

{\large\bf Assignment \# 6}
\smallskip

{\large\bf John Joseph}
\end{center}

{\bf Problem 1}
\smallskip

In order to use the mediant bisection, let us denote the lower and upper bounds of our enclosing interval as $\frac{a}{b}$ and $\frac{c}{d}$, respectively. Therefore the mediant $m=\frac{a+c}{b+d}$. We will use this method to find a rational fraction to approximate $\sqrt{2}$ using $\frac{a}{b}=\frac{7}{5}$, $\frac{c}{d}=\frac{17}{12}$. 

\begin{eqnarray*}
m_1=\frac{7+17}{5+12}=\frac{24}{17}=1.411764 <  \sqrt{2} \\
\frac{a}{b} \leftarrow \frac{24}{17} \\
m_2=\frac{24+17}{29+12}=\frac{41}{29}=1.413793 \\
|\sqrt{2}-1.413793|=0.000420
\end{eqnarray*}

{\bf Problem 2}
\smallskip

We will confirm the convergence of an iterative function to a target of $\sqrt{2}$ and find its order of convergence. 

\begin{eqnarray*}
x_{i+1}=\frac{x_i+2}{x_i+1}\\
x_0=0 \\
x_1=2\\
x_2=\frac{4}{3}\\
\end{eqnarray*}

To determine the order of convergence we will compare the difference of the first iteration from the exact solution. 

\begin{eqnarray*}
x_1-\sqrt{2}=\frac{x_i+2}{x_i+1}-\sqrt{2}=\frac{x_0+2-\sqrt{2}(x_0+1)}{x_0+1} \\
x_1-\sqrt{2}=(x_0-\sqrt{2})(1-\sqrt{2})
\end{eqnarray*}

The fact that $|1-\sqrt{2}| < 1$ implies the solution converges (each iteration has less error than the last), and the difference term is linear in power. This tells us that the order of convergence is linear. 
\bigskip

{\bf Problem  3}
\smallskip

We will use Halley's method to find one of the roots of the function $f(x)=Ax+Bx^2$. We will confirm a root of $x=0$, although there should be another root at $x=-\frac{A}{B}$. Using the methods used in Problem 2 and knowledge of a root at $x=0$, we can do the following:

\begin{eqnarray*}
x_{i+1}=x_i- \Big{(} \frac{2f'(x_i)}{2f'(x_i)^2-f(x_i)f''(x_i)} \Big{)} f(x_i) \\
f'(x)=A+2Bx \\
f''(x)=2B \\
x_1-0=x_0-\Big{(} \frac{2f'(x_0)}{2f'(x_0)^2-f(x_0)f''(x_0)} \Big{)} f(x_0) -0 \\
x_1-0=x_0-0-2 \Big{(} \frac{A+2Bx_0}{(A+2Bx_0)^2-(Ax_0+Bx_0^2)(B)} \Big{)}(Ax_0+Bx_0^2) \\
x_1=x_0^3 \Big{(} \frac{B^2}{A^2+3B^2x_0^2+3ABx_0} \Big{)}
\end{eqnarray*} 

This shows that the order of convergence is cubic (difference is cubed). As we near the root $x=0$ the difference coefficient goes to $\frac{B^2}{A^2}$. I'm still not sure about the other root, although I assume cubic convergence can be found for it as well. 
\bigskip

{\bf Problem  4}
\smallskip

We are asked to find the roots of the function $f(x)=\cos{x}-x$ using the fixed point and Newton-Raphson methods with a starting value of $x_0=0.6$. 
\smallskip

Fixed Point:

\begin{eqnarray*}
x=\cos{x} \\
x_1=\cos{x_0}=0.825335 \\
x_2=\cos{x_1}=0.678310 \\
x_3=\cos{x_2}=0.778634 \\
f(x_3)=-0.066761
\end{eqnarray*}

Newton Raphson: 

\begin{eqnarray*}
f'(x)=-\sin{x}-1 \\
x=x-\frac{f(x)}{f'(x)} \\
x_1=x_0-\frac{f(x_0)}{f'(x_0)} = 0.744017 \\
x_2=x_1-\frac{f(x_0)}{f'(x_0)} = 0.739091 \\
x_3=x_2-\frac{f(x_2)}{f'(x_2)} = 0.739085 \\
f(x_3) = -1.055212 \times 10^{-11}
\end{eqnarray*}

It is clear to see that Newton-Raphson is superior. 
\bigskip

{\bf Problem  5}
\smallskip

We will fix the variable $m$ in the following modification of the Newton-Raphson method

\begin{equation*}
x_{i+1}=x_i-m \frac{f(x_i)}{f'(x_i)}
\end{equation*}

so as to maximize the rate of convergence. As we've seen, the order at which the approximation converges can be obtained by looking at how the error value behaves. For the function $f(x)=a_3x^3+a_4x^4$, the roots of which are $x=0$, $x=-\frac{a_3}{a_4}$, we will check the error for the first root at the first iteration. 

\begin{eqnarray*}
x_1-0=x_0-0-m \Big{(} \frac{a_3x_0^3+a_4x_0^4}{3a_3x_0^2+4a_4x_0^3} \Big{)} \\
x_1=\frac{x_0^2(a_3x_0(3-m)+a_4x_0^2(4-m))}{x_0^2(3a_3+4a_4x_0)}  \\
x_1= \frac{a_3x_0(3-m)+a_4x_0^2(4-m)}{3a_3+4a_4x_0}  \\
\end{eqnarray*}

In the numerator, we see that the linear $x_0$ term prevents the approximation from converging quadratically. By setting $m=3$ we remove this term and restore quadratic convergence. 

\begin{equation*}
x_{i+1}=x_i-3 \frac{f(x_i)}{f'(x_i)}
\end{equation*}

{\bf Problem 6}
\smallskip

In order to solve a system of equations using the Newton-Raphson method, we will make some changes in notation. $X$ represents the vector $[x,y]^T$, $F=[f(x,y),g(x,y)]^T$. $J$ is the Jacobian matrix, which we will use in place of the derivative $f'$ in the original formula. Therefore, a Newton-Raphson iteration becomes

\begin{equation*}
X_{i+1}=X_i-J_i^{-1}F_i \\
\end{equation*}

Solving this system for $f(x)=x^2-\cos{y}$, $g(x)=\sin{x}+x^2+y^3$ means that

\begin{equation*}
J=
 \begin{bmatrix}
	2x & \sin{y}  \\[0.2em]
	\cos{x}+2x & 3y^2  \\[0.2em]
     \end{bmatrix}
\end{equation*}

starting from $X_0=[1.5,-1.25]^T$ and performing one iteration yields that 

\begin{equation*}
F_0=\begin{bmatrix} 1.93467763 \\[0.2em] 1.29436998 \\[0.2em] \end{bmatrix}
\end{equation*}

\begin{eqnarray*}
X_1=\begin{bmatrix} 1.5 \\[0.2em] -1.25 \\[0.2em] \end{bmatrix}-
 \begin{bmatrix}
	3 & \sin{(-1.25)}  \\[0.2em]
	\cos{(1.5)}+3 & 3(1.25)^2  \\[0.2em]
     \end{bmatrix}^{-1}
\begin{bmatrix} 1.93467763 \\[0.2em] 1.29436998 \\[0.2em] \end{bmatrix} \\
X_1= \begin{bmatrix} 0.89345046 \\[0.2em] -1.12878736 \\[0.2em] \end{bmatrix}
\end{eqnarray*}
\begin{equation*}
F_1=\begin{bmatrix} 0.37049751 \\[0.2em] 0.13923588 \\[0.2em] \end{bmatrix}
\end{equation*}

\vfil\eject

{\bf Problem 7}

Two relaxation iterations will be computer for this system of equations. In the first, we do the following:

\begin{eqnarray*}
2x_1=-3 \\
x_1=-\frac{3}{2} \\
\frac{3}{2}+2x_2=2 \\
x_2=\frac{1}{4} \\
-\frac{1}{4}+x_3=0 \\
x_3=\frac{1}{4}
\end{eqnarray*}

So in the first sweep $x_1=-1.5$, $x_2=x_3=0.25$. In the second sweep

\begin{eqnarray*}
2x_1-\frac{1}{4}=-3 \\
x_1=-\frac{11}{8} \\
\frac{11}{8}+2x_2-\frac{1}{4}=2 \\
x_2=\frac{7}{16} \\
-\frac{7}{16}+x_3=0 \\
x_3=\frac{7}{16}
\end{eqnarray*}

Now $x_1=1.375$, $x_2=x_3=0.4375$. Slowly but surely we are getting to the expected values of $x_1=1$, $x_2=x_3=1$

\end{document}
