\documentclass[a4paper,12pt]{article}
\parindent 0pt
\parskip 1mm
\usepackage{amsmath}
\usepackage[dvips]{epsfig}

\begin{document}

\begin{center}

{\Large\bf MA 555 - Numerical Analysis}

\bigskip

{\large\bf Assignment \# 3}
\smallskip

{\large\bf John Joseph}
\end{center}

{\bf Problem 1}
\smallskip

We can approximate the function $f(x)=\sqrt{x}$ with a linear Lagrange interpolant $\widetilde{y}(x)$ between some interval $0 \leq x \leq h$ like so:

\begin{equation*}
\widetilde{y}(x) = L_1(x)f(x_1)+L_2(x)f(x_2)
= \Big{[} \frac{x-x_2}{x_1-x_2} \Big{]} \sqrt{x_1} + \Big{[} \frac{x-x_1}{x_2-x_1} \Big{]} \sqrt{x_2} \Big{|}_{x_1=0,x_2=h} 
\end{equation*}
\begin{equation*}
\widetilde{y}(x) = 0+ \big(\frac{x}{h} \big) \sqrt{h} = \frac{x \sqrt{h}}{h}
\end{equation*}

Our error function $e(x)=|f(x)-\widetilde{y}(x)|$ has a maximum which can be found as follows:
\begin{eqnarray*}
e(x)=\sqrt{x}- \frac{x \sqrt{h}}{h}\\
e'(x)=\frac{1}{2}x^{-\frac{1}{2}}-\frac{\sqrt{h}}{h}=0\\
x=\frac{h}{4}\\
e'(x < \frac{h}{4}) > 0; 
e'(x > \frac{h}{4}) < 0\\
e_{max}(x)=e(\frac{h}{4})=|\sqrt{\frac{h}{4}}-\frac{\sqrt{h}}{h} \frac{h}{4}| \\
e_{max}(x)=\frac{\sqrt{h}}{4} \\
\end{eqnarray*}

{\bf Problem 2}
\smallskip

We can find the parabolic Lagrange interpolant to the function $f(x)=x^3$ using the three points $x_1=1,x_2=2,x_3=3$ (solved with Mathematica). 
\begin{equation*}
\begin{split}
\widetilde{y}(x) &=\Big{[} \frac{(x-x_2)(x-x_3}{(x_1-x_2)(x_1-x_3)} \Big{]}f(x_1)+\Big{[} \frac{(x-x_1)(x-x_3}{(x_2-x_1)(x_2-x_3)} \Big{]}f(x_2) \\ 
 &+\Big{[} \frac{(x-x_1)(x-x_2}{(x_3-x_2)(x_3-x_1)} \Big{]}f(x_3) = 6x^2-11x+6
\end{split}
\end{equation*}

We can find the maximum error value by using the same method used above:

\begin{eqnarray*}
e(x)=|x^3-(6x^2-11x+6)|\\
e'(x)=3x^2-12x+11=0\\
x=\frac{1}{3}(6 \pm \sqrt{3})\\
e_{max}(x)=e(\frac{1}{3}(6 \pm \sqrt{3}))=0.3849 \\
\end{eqnarray*}

{\bf Problem  3}
\smallskip

We can improve our approximations drastically by mapping the Chebyshev roots to be bound within our interval. To approximate the function $f(x)=\sqrt{1+x}$ linearly we need the two roots $\xi_{1,2}=\pm \frac{\sqrt{2}}{2}$ mapped to our interval $0 \leq x \leq \frac{1}{4}$, which we will call $c_{1,2}$. 

\begin{eqnarray*}
\xi \rightarrow x: x(\xi)=\frac{x_1}{2}[1-\xi]+\frac{x_2}{2}[1+\xi] \Big{|}_{x_1=0,x_2=\frac{1}{4}} \\
x(\xi_{1,2})=c_{1,2}=\frac{1}{8}(1 \pm \frac{\sqrt{2}}{2}
\end{eqnarray*}

Using these two Chebyshev roots, we can write our Lagrange Interpolant as follows:

\begin{equation*}
\widetilde{y}(x)=\Big{[} \frac{x-c_2}{c_1-c_2} \Big{]} f(c_1) + \Big{[} \frac{x-x_1}{c_2-c_1} \Big{]} f(c_2) = 1.00087 + 0.471769 x
\end{equation*}

{\bf Problem  4}
\smallskip

This problem is similar to \#3 except we now have three Chebyshev roots and are dealing with the function $f(x)=\frac{1}{1+x}$. We map the $3^{rd}$ degree Chebyshev roots, $\xi_{1,3}=\pm \frac{\sqrt{3}}{2}$, $\xi_2=0$, to our interval $0 \leq x \leq \frac{1}{2}$:

\begin{eqnarray*}
\xi \rightarrow x: x(\xi)=\frac{x_1}{2}[1-\xi]+\frac{x_3}{2}[1+\xi] \Big{|}_{x_1=0,x_2=\frac{1}{4}} \\
x(\xi_{1,3})=c_{1,3}= \frac{1}{4}(1 \pm \frac{\sqrt{3}}{2})\\
x(\xi_2)=c_2=\frac{1}{4}
\end{eqnarray*}

Having mapped the roots ont our interval, we use the Lagrange interpolants to find the parabola that passes through all three:

\begin{equation*}
\begin{split}
\widetilde{y}(x) &=\Big{[} \frac{(x-c_2)(x-c_3}{(c_1-c_2)(c_1-c_3)} \Big{]}f(c_1)+\Big{[} \frac{(x-c_1)(x-c_3}{(c_2-c_1)(c_2-c_3)} \Big{]}f(c_2) \\ 
 &+\Big{[} \frac{(x-c_1)(x-c_2}{(c_3-c_2)(c_3-c_1)} \Big{]}f(c_3) = \frac{4}{99}(64x^2+16x+25)
\end{split}
\end{equation*}

{\bf Problem  5}
\smallskip

We are asked to find the Bezier curve going through the points (0,0) and (2,0) with tangent vectors $<0,1>$ and $<0,-1>$, respectively. 

\begin{eqnarray*}
x(t)=a_0+a_1t+a_2t^2+a_3t^3 \\
x(0)=0=a_0, x(1)=2=a_0+a_1+a_2+a_3 \\
x'(t)=a_1+2a_2t+3a_3t^2 \\
x'(0)=a_1=0 \\
x'(1)=a_1+2a_2+3a_3=0 \\
\\
y(t)=b_0+b_1t+b_2t^2+b_3t^3 \\
y(0)=0=b_0, y(1)=0=b_0+b_1+b_2+b_3 \\
y'(t)=b_1+2b_2t+3b_3t^2 \\
y'(0)=b_1=1 \\
y'(1)=b_1+2b_2+3b_3=-1
\end{eqnarray*}

From this we have 8 equations with which we can solve for our 8 unknowns. We write the system as a Matrix and solve it in Mathematica. 

\begin{equation*}
 \begin{bmatrix}
	1 & 0 & 0 & 0 & 0 & 0 & 0 & 0           \\[0.2em]
	1 & 1 & 1 & 1 & 0 & 0 & 0 & 0           \\[0.2em]
	0 & 1 & 0 & 0 & 0 & 0 & 0 & 0           \\[0.2em]
	0 & 1 & 2 & 3 & 0 & 0 & 0 & 0           \\[0.2em]
	0 & 0 & 0 & 0 & 1 & 0 & 0 & 0           \\[0.2em]
	0 & 0 & 0 & 0 & 1 & 1 & 1 & 1           \\[0.2em]
	0 & 0 & 0 & 0 & 0 & 1 & 0 & 0           \\[0.2em]
	0 & 0 & 0 & 0 & 0 & 1 & 2 & 3           \\[0.2em]
     \end{bmatrix}
 \begin{bmatrix}
	a_0           \\[0.2em]
	a_1          \\[0.2em]
	a_2          \\[0.2em]
	a_3           \\[0.2em]
	b_0           \\[0.2em]
	b_1          \\[0.2em]
	b_2          \\[0.2em]
	b_3           \\[0.2em]
     \end{bmatrix}
=
 \begin{bmatrix}
	0           \\[0.2em]
	2          \\[0.2em]
	0          \\[0.2em]
	0           \\[0.2em]
	0           \\[0.2em]
	0          \\[0.2em]
	1          \\[0.2em]
	-1           \\[0.2em]
     \end{bmatrix}
\end{equation*}
From this we can solve for the coeffecient vector and see that 

\begin{eqnarray*}
x(t)=6t^2-4t^3 \\
y(t)=t-t^2
\end{eqnarray*}

{\bf Problem 6}
\smallskip

We can join the two splines by imposing certain continuity conditions, namely:

\begin{eqnarray*}
y_1=s_1(-1)=a_0-a_1+a_2-a_3 \\
s_1''(-1)=0=a_2-3a_3 \\
s_1(0)=s_2(0)=y_2=a_0=b_0 \\
s_1'(0)=s_2'(0)=a_1=b_1 \\
s_1''(0)=s_2''(0)=a_2=b_2 \\
y_3=b_0+b_1+b_2+b_3 \\ 
s_2''(1)=0=b_2+3b_3 \\ 
\end{eqnarray*}

We again have 8 equations with 8 unknowns, which we can use to create a matrix representing our system of equations. We solve the matrix in Mathematica. 

\begin{equation*}
 \begin{bmatrix}
	1 & -1 & 1 & -1 & 0 & 0 & 0 & 0           \\[0.2em]
	0 & 0 & 1 & -3 & 0 & 0 & 0 & 0           \\[0.2em]
	1 & 0 & 0 & 0 & 0 & 0 & 0 & 0           \\[0.2em]
	0 & 1 & 0 & 0 & 0 & -1 & 0 & 0           \\[0.2em]
	0 & 0 & 0 & 0 & 1 & 0 & 0 & 0           \\[0.2em]
	0 & 0 & 0 & 0 & 1 & 1 & 1 & 1           \\[0.2em]
	0 & 0 & 0 & 0 & 0 & 0 & 1 & 3           \\[0.2em]
	0 & 0 & 1 & 0 & 0 & 0 & -1 & 0           \\[0.2em]
     \end{bmatrix}
 \begin{bmatrix}
	a_0           \\[0.2em]
	a_1          \\[0.2em]
	a_2          \\[0.2em]
	a_3           \\[0.2em]
	b_0           \\[0.2em]
	b_1          \\[0.2em]
	b_2          \\[0.2em]
	b_3           \\[0.2em]
     \end{bmatrix}
=
 \begin{bmatrix}
	y_1           \\[0.2em]
	0          \\[0.2em]
	y_2          \\[0.2em]
	0           \\[0.2em]
	y_2           \\[0.2em]
	y_3         \\[0.2em]
	0          \\[0.2em]
	0           \\[0.2em]
     \end{bmatrix}
\end{equation*}

From which we get

\begin{eqnarray*}
a_0=b_0=y2 \\
a_1=b_1=\frac{1}{2}(y_3-y_1) \\
a_2=b_2=\frac{3}{4}(y_1+y_3-2y_2) \\
a_3=\frac{1}{4}(y_1+y_3-2y_2) \\
b_3=-\frac{1}{4}(y_1+y_3-2y_2) \\
\end{eqnarray*}

{\bf Problem 7}

This problem is like problem 6, except our curve is quadratic and we have real values for our variables. 

\begin{eqnarray*}
s_1(-1)=0=a_0-a_1+a_2 \\
s_2(2)=0=b_0+2b_1+4b_2 \\
s_1(0)=s_2(0)=1=a_0=b_0
s_1'(0)=s_2'(0)=\frac{1}{4}=a_1=b_1 \\
0=1-\frac{1}{4}+a_2 \\
a_2=-\frac{3}{4} \\
0 = 1+\frac{1}{2}+4b_2 \\
b_2=-\frac{3}{8} \\ 
s_1(x)=1+\frac{1}{4}x-\frac{3}{4}x^2 \\
s_2(x)=1+\frac{1}{4}x-\frac{3}{8}x^2 
\end{eqnarray*}

\end{document}
